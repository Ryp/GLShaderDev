\documentclass[10pt, a4paper]{article} % Default font size is 12pt, it can be changed here

\usepackage[T1]{fontenc}
\usepackage[french]{babel}

\usepackage{geometry} % Required to change the page size to A4
\usepackage{graphicx} % Required for including pictures
\usepackage{float} % Allows putting an [H] in \begin{figure} to specify the exact location of the figure

\newcommand{\HRule}{\rule{\linewidth}{0.5mm}} % Defines a new command for the horizontal lines, change thickness here

\begin{document}
\pagenumbering{gobble}
%----------------------------------------------------------------------------------------

\begin{center}
\HRule \\[0.2cm]
{ \huge \bfseries Fiche Projet - GLShaderDev \\
  \small Environnement de d�veloppement pour shaders OpenGL
}\\[0.2cm]
\HRule
\end{center}

\section*{Introduction}
GLShaderDev est un IDE Open source pour GLSL (OpenGL Shading Language). \\
Ecris en C++, utilisant Qt, il permettra de compiler automatiquement le code GLSL et d'en avoir la visualisation directe dans l'�diteur.
Un log de compilation intelligent sera integr� avec possibilit� d'�tre renvoy� aux lignes de code posant probl�me, et l'�diteur proposera la coloration ainsi que la completion. Toute la difficult� sera de permettre � l'utilisateur de travailler sur ses shaders sans �crire le moindre appel � l'API OpenGL.

\section*{Contexte}
OpenGL est un outil qui attire de plus en plus l'attention des d�veloppeurs de jeux, mais dispose de relativement peu d'outils performants face � DirectX, son concurrent de Microsoft.
Cet IDE est une tentative pour combler le manque d'outils de d�veloppement de shaders, et de pourvoir � la communaut� OpenGL un outil fiable et � jour.

\section*{Equipe}
\noindent
\textbf{Thibault Schueller:} Int�gration d'OpenGL et parsing GLSL \\
\textbf{St�phane Nguyen:} Interface utilisateur Qt

\section*{Partenaires}
Aucun partenaire pour l'instant, mais avoir un contact avec des soci�t�s comme ATI ou Nvidia pourrait �tre extr�mement pertinent.
Le support de la communaut� OpenGL pourrait aussi �tre tr�s int�ressant pour ce projet.

\section*{Objectifs}
Le d�p�t sera h�berg� sur github, pour permettre l'interaction avec la communaut�. \\
Un objectif de 1000 t�l�chargements sur la page github nous parait pertinent.

\section*{Planning g�n�ral}
\noindent
08/12/2013: 1$^{er}$ prototype utilisable avec support de la compilation automatique \\
08/02/2014: Completion et coloration syntaxique \\
08/05/2014: Integration OpenGL compl�te \\
08/06/2014: Fin du PFA

%----------------------------------------------------------------------------------------
\end{document}
